\documentclass{article}
\usepackage[utf8]{inputenc}
\usepackage[MeX,plmath]{polski} 
\usepackage{amsmath}
\usepackage{amsfonts}

\begin{document}
\title{Rozkład Studenta}
\maketitle

\section*{Wprowadzenie}
Rozkład Studenta (rozkład t lub rozkład t-Studenta) – ciągły rozkład prawdopodobieństwa stosowany często w statystyce w procedurach testowania hipotez statystycznych i przy ocenie niepewności pomiaru. Przy opracowaniu wyników pomiarów często powstaje zagadnienie oszacowania przedziału, w którym leży, z określonym prawdopodobieństwem, rzeczywista wartość mierzona, jeśli dysponujemy tylko wynikami n pomiarów, dla których możemy wyznaczyć takie parametry, jak średnia i odchylenie standardowe lub wariancja, nie znamy natomiast odchylenia standardowego w populacji. Zagadnienie to rozwiązał w 1908 r. William Sealy Gosset (pseudonim Student) podając funkcję zależną od wyników pomiarów, a niezależną od sigmy.

\section*{Definicja}
Rozkład Studenta z n stopniami swobody jest rozkładem zmiennej losowej T postaci: 
\[ T={\frac {U}{\sqrt {Z}}}{\sqrt {n}} \]

\section*{Gęstość prawdopodobieństwa}
Zmienna losowa T określona powyżej ma gęstość prawdopodobieństwa opisaną wzorem: 
$$ f(t,n)={\frac {\Gamma ({\frac {n+1}{2}})}{\Gamma ({\frac {n}{2}}){\sqrt {n\pi }}}}\left(1+{\frac {t^{2}}{n}}\right)^{-{\frac {n+1}{2}}} $$

Dowód. Gęstość wyraża się wzorem
\[ f_{Y}(y)={\frac {2^{1-{\frac {n}{2}}}y^{n-1}e^{-{\frac {y^{2}}{2}}}}{\Gamma ({\frac {n}{2}})}} \]

Rozważmy zmienną
$$ X={\frac  {1}{{\sqrt  {n}}}}Y $$

Wówczas
$$ {\frac  {\partial Y}{\partial X}}={\sqrt  n} $$
a zatem całkując przez podstawienie obserwujemy, że 
\[ {\begin{aligned}f_{X}(x)&=f_{Y}({\sqrt {n}}x){\Big |}{\frac {\partial Y}{\partial X}}{\Big |}\\&={\frac {2^{1-{\frac {n}{2}}}}{\Gamma \left({\frac {n}{2}}\right)}}({\sqrt {n}}x)^{n-1}e^{-{\frac {({\sqrt {n}}x)^{2}}{2}}}{\sqrt {n}}\\&={\frac {2^{1-{\frac {n}{2}}}}{\Gamma \left({\frac {n}{2}}\right)}}n^{\frac {n}{2}}x^{n-1}e^{-{\frac {n}{2}}x^{2}}.\end{aligned}} \]

Zmienna T ma zatem rozkład U/X. Jej gęstość jest więc postaci 
\begin{displaymath}
{\begin{aligned}f_{T}(t)&=\int \limits _{-\infty }^{\infty }|x|f_{U}(xt)f_{X}(x)\,\mathrm {d} x=\int \limits _{0}^{\infty }xf_{U}(xt)f_{X}(x)\,\mathrm {d} x\\&=\int \limits _{0}^{\infty }x{\frac {1}{\sqrt {2\pi }}}e^{-{\frac {(xt)^{2}}{2}}}{\frac {2^{1-{\frac {n}{2}}}}{\Gamma \left({\frac {n}{2}}\right)}}n^{\frac {n}{2}}x^{n-1}e^{-{\frac {n}{2}}x^{2}}\,\mathrm {d} x\\&={\frac {n^{\frac {n}{2}}}{\sqrt {2\pi }}}{\frac {2^{1-{\frac {n}{2}}}}{\Gamma \left({\frac {n}{2}}\right)}}\int \limits _{0}^{\infty }x^{n}e^{-{\frac {1}{2}}(n+t^{2})x^{2}}\,\mathrm {d} x.\end{aligned}}
\end{displaymath}

Powyższa całka przyjmuje postać
\begin{equation*}
\int \limits _{0}^{\infty }x^{n}e^{-{\frac {1}{2}}(n+t^{2})m}{\frac {\mathrm {d} m}{2x}}={\frac {1}{2}}\int \limits _{0}^{\infty }m^{\frac {n-1}{2}}e^{-{\frac {1}{2}}(n+t^{2})m}\mathrm {d} m
\end{equation*}

Oznacza to, że
$$ k-1={\frac {n-1}{2}}\Rightarrow k^{*}={\frac {n+1}{2}},\qquad {\frac {1}{\theta }}={\frac {1}{2}}(n+t^{2})\Rightarrow \theta ^{*}={\frac {2}{(n+t^{2})}} $$

Ostatecznie
$$ f_{T}(t)={\frac {1}{\sqrt {2\pi }}}{\frac {2^{1-{\frac {n}{2}}}}{\Gamma \left({\frac {n}{2}}\right)}}n^{\frac {n}{2}}2^{\frac {n-1}{2}}n^{-{\frac {n+1}{2}}}\Gamma \left({\frac {n+1}{2}}\right)\left(1+{\frac {t^{2}}{n}}\right)^{-{\frac {1}{2}}(n+1)}={\frac {\Gamma [(n+1)/2]}{{\sqrt {n\pi }}\Gamma (n/2)}}\left(1+{\frac {t^{2}}{n}}\right)^{-{\frac {1}{2}}(n+1)} $$

\end{document}