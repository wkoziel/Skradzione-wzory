\documentclass{article}
\usepackage[utf8]{inputenc}
\usepackage[MeX,plmath]{polski} 
\usepackage{amsmath}
\usepackage{amsfonts}

\begin{document}
\title{Małe twierdzenie Fermata}
\maketitle

\section*{Wprowadzenie}
Małe twierdzenie Fermata (MTF) – twierdzenie teorii liczb sformułowane (bez dowodu) przez francuskiego matematyka Pierre’a de Fermata. Twierdzenie jest podstawą dla testu pierwszości Fermata. Poniżej każdego sformułowania twierdzenia znajduje się zapis w arytmetyce modularnej.

Małe twierdzenie Fermata:

jeżeli p jest liczbą pierwszą, to dla dowolnej liczby całkowitej a, liczba $a^p - a$ jest podzielna przez p.
$$ a^{p}-a\equiv 0{\pmod {p}} $$

lub inaczej

jeśli p jest liczbą pierwszą, a a jest taką liczbą całkowitą, że liczby a p są względnie pierwsze, to \(a^{p-1}-1\) dzieli się przez p. Innymi słowy, 

\begin{equation}
	a^{p-1}-1\equiv 0{\pmod {p}}
\end{equation}
albo

\[ a^{p-1}\equiv 1{\pmod {p}} \]

\section*{Dowody}
\subsection*{Dowód z twierdzeniem Eulera}
Jeżeli p  jest taką liczbą pierwszą, która nie dzieli a, to p jest względnie pierwsze z a, a więc w myśl twierdzenia Eulera o liczbach względnie pierwszych teza twierdzenia jest prawdziwa. 

\subsection*{Dowód kombinatoryczny}
Bez straty ogólności można założyć, że a jest liczbą naturalną. Rozpatrzmy wszystkie możliwe kolorowania koła podzielonego na p części za pomocą a kolorów. Kolorowania, które możemy na siebie nałożyć po obróceniu, liczymy jako dwa różne. Wszystkich kolorowań jest a do p.

Wszystkie kolorowania, w których wykorzystaliśmy co najmniej dwa kolory możemy obracać tak, że otrzymamy zestawy po p parami różnych kolorowań, które są swoimi obrotami (przykładowe cztery z pewnego zestawu dla p=7, a=3 są przedstawione na rysunku). Jeżeli w pewnym zestawie utworzonym w ten sposób wystąpiłyby takie same kolorowania, to oznaczałoby to, że kąt pełny jest wielokrotnością pewnego kąta, o który trzeba obrócić jedno z tych kolorowań, aby otrzymać drugie. W przypadku, gdy wykorzystaliśmy jeden kolor, nie jest to możliwe. Zatem: 

liczba wszystkich kolorowań jest iloczynem p i liczby zestawów po p kolorowań + liczba kolorowań jednokolorowych

\begin{displaymath}
	a^{p}=pn+a
\end{displaymath} 
gdzie n jest pewną liczbą naturalną.

Kolorów jest a, więc kolorowań jednokolorowych też jest a. Wynika stąd, że p dzieli liczbę a do p - a.

\subsection*{Dowód indukcyjny}
Załóżmy, że a jest liczbą naturalną. Twierdzenie to jest prawdziwe, gdy a = 1. Zatem załóżmy indukcyjnie, że:

\begin{equation}
	a^{p}\equiv a{\pmod {p}}
\end{equation}

Wówczas:

$$(a+1)^{p}=\sum _{k=0}^{p}{p \choose k}a^{k}=a^{p}+a^{0}+\sum _{k=1}^{p-1}{p \choose k}a^{k}$$

Ponieważ

$${p \choose k}={\frac {p(p-1)(p-2)\cdots (p-k+1)}{k!}}$$

więc dla 0 mniejsze od k mniejsze od p żaden z czynników k! nie jest równy p, dlatego $p \choose k$ jest wielokrotnością p.

\end{document}
