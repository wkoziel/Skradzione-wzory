\documentclass{article}
\usepackage[utf8]{inputenc}
\usepackage[MeX,plmath]{polski} 
\usepackage{amsmath}
\usepackage{amsfonts}

\begin{document}
\title{Trójki pitagorejskie}
\maketitle

\section*{Wstęp}
Trójka pitagorejska (albo liczby pitagorejskie) – trzy liczby całkowite dodatnie a,b,c spełniające tzw. równanie Pitagorasa: 

$$a^{2}+b^{2}=c^{2}.$$

Ich nazwa pochodzi od twierdzenia Pitagorasa, na mocy którego boki trójkąta prostokątnego spełniają powyższą zależność. W poniższej tabeli przedstawiono kilka pierwszych (względem krótszej przyprostokątnej) trójek pitagorejskich: 

\begin{center}
	
\begin{tabular}{ c c c }
	a & b & c \\
	3 & 4 & 5 \\ 
	5 & 12 & 13 \\  
	6 & 8 & 10 \\
	7 & 24 & 25 \\
	8 & 15 & 17 \\
	9 & 12 & 15 \\
	12 & 16 & 20    
\end{tabular}

\end{center}


\section*{Własności}
Jeżeli trójka (a,b,c) jest pitagorejska, to jest nią też (da,db,dc), dla dowolnej liczby całkowitej dodatniej d. Trójkę pitagorejską nazywamy pierwotną, jeśli a, b i c nie mają wspólnego dzielnika większego od 1. Zatem z każdej trójki pitagorejskiej możemy uzyskać pierwotną przez podzielenie jej przez największy wspólny dzielnik i dowolną trójkę pitagorejską możemy otrzymać z pierwotnej przez pomnożenie jej wszystkich trzech elementów przez odpowiednią tę samą liczbę całkowitą dodatnią. 

Jeśli m większe od n są liczbami całkowitymi dodatnimi, to 

\begin{equation}
	a=m^{2}-n^{2},
\end{equation}

\begin{equation*}
	b=2\cdot m\cdot n,
\end{equation*}

\begin{displaymath}
	c=m^{2}+n^{2}
\end{displaymath}

\section*{Rozwiązanie elementarne}
Kwadrat nieparzystej liczby naturalnej przy dzieleniu przez 8 daje resztę 1. Zatem suma kwadratów dwóch dowolnych liczb nieparzystych daje resztę 2 z dzielenia przez 8. Z drugiej strony, kwadrat dowolnej liczby naturalnej daje przy dzieleniu przez 8 jedną z reszt 0, 1, 4. Zatem suma dwóch kwadratów nieparzystych liczb naturalnych nigdy nie jest kwadratem.

Niech a,b,c będą liczbami naturalnymi, spełniającymi równanie:

\[a^{2}+b^{2}=c^{2}\]

Zatem co najmniej jedna z liczb a,b, jest parzysta. Przy założeniu, że a,b są względnie pierwsze, jedna z liczb a,b, powiedzmy a, jest nieparzysta, a b jest parzysta. Zatem c jest nieparzysta i względnie pierwsza zarówno z liczbą a, jak i z liczbą b.

Każdy wspólny dzielnik liczb naturalnych (c-a)/2 oraz ( c + a ) / 2 jest też dzielnikiem ich sumy, równej c, oraz ich różnicy, równej a, jest więc równy 1, a liczby ( c - a ) / 2 oraz ( c + a ) / 2 są względnie pierwsze. 



Równanie (1) ma dokładnie te same rozwiązania (a,b,c), co równanie:

$$ {\frac {c-a}{2}}\cdot {\frac {c+a}{2}}=\left({\frac {b}{2}}\right)^{2} $$

Ponieważ liczby (c-a)/2 oraz (c+a)/2 są względnie pierwsze, to są pełnymi kwadratami pewnych liczb naturalnych m oraz n:

\begin{equation}
	{\frac {c+a}{2}}=m^{2}
\end{equation}

skąd:

$$ a=m^{2}-n^{2} $$

$$ b=2\cdot m\cdot n $$

$$ c=m^{2}+n^{2} $$

\end{document}