\documentclass{article}
\usepackage[utf8]{inputenc}
\usepackage[MeX,plmath]{polski} 
\usepackage{amsmath}
\usepackage{amsfonts}

\begin{document}
\title{Twierdzenie Rolle’a}
\maketitle

\section*{Wprowadzenie}
Twierdzenie Rolle’a – twierdzenie klasycznej analizy matematycznej mówiące, że funkcja różniczkowalna przyjmująca równe wartości w dwóch różnych punktach ma pomiędzy nimi punkt stacjonarny, tzn. punkt, w którym nachylenie prostej stycznej do wykresu funkcji względem osi OX jest równe zeru.

Twierdzenie to opublikował (dla wielomianów) francuski matematyk Michel Rolle w 1691. W innej postaci znane ono było w 1150 roku hinduskiemu matematykowi Bhaskarze. 

\section*{Wersja standardowa}

Niech f będzie ciągłą funkcją rzeczywistą określoną na przedziale domkniętym [a,b], różniczkowalną na przedziale otwartym (a, b). Wówczas jeżeli f(a) = f(b), to istnieje taki punkt c należący do przedziału otwartego (a, b), że 

\begin{equation}
	f'(c)=0
\end{equation}

Z tej wersji twierdzenia Rolle’a korzysta się przy dowodzie twierdzenia Lagrange’a o wartości średniej, którego twierdzenie Rolle’a jest przypadkiem szczególnym. 

\section*{Dowód}
Jeżeli f = const, to f'(c) = 0 dla każdego c należącego do (a, b). Gdy f nie jest tożsamościowo równa stałej, to istnieje taki punkt x należący do (a, b), dla którego zachodzi 

\begin{displaymath}
	f(x) > f(a) = f(b)
\end{displaymath}

lub

\begin{equation*}
	f(x) < f(a) = f(b)
\end{equation*}

Przypuśćmy, że zachodzi pierwszy przypadek, tzn. dla pewnego argumentu wartość funkcji jest większa od f(a) = f(b); rozumowanie w drugim przypadku jest analogiczne (wówczas trzeba rozważać wartość najmniejszą zamiast największej).

Określona na przedziale zwartym [a,b] funkcja ciągła f na mocy twierdzenia Weierstrassa przyjmuje wartość największą, tzn. istnieje taki punkt $c \in [a, b]$, że

\[f(c) = \sup f(x)\] 

dla x należącego do [a,b].

Z założenia, że istnieje wartość większa od f(a) = f(b) wynika, że $a \neq c \neq b$ , tzn. c należy do (a, b). Warunkiem koniecznym istnienia ekstremum globalnego funkcji f w c jest znikanie pochodnej w tym punkcie, co dowodzi tezy. 

\section*{Uogólnienia}
Niech h := b - a będzie rzeczywistą liczbą dodatnią, a x := a, wtedy x + h = b. Punkt c należący do (a,b) można zapisać jako x + 0 h, gdzie 0 należy do (0,1).

Przy takich oznaczeniach twierdzenie Rolle’a ma postać: 

Jeśli

\begin{equation*}
	 f(x+h)=f(x)
\end{equation*}

to istnieje punkt x + 0h, dla którego

\[f'(x+\theta h)h=0\]

Rezygnacja z warunku \(f(x)=f(x+h)\), prowadzi do ogólniejszego twierdzenia Lagrange’a:

Istnieje taki punkt x + 0h, który spełnia tożsamość

\begin{equation*}
	f(x + h) = f(x) + f'(x + \theta h) h
\end{equation*}

\end{document}