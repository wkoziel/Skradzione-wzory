\documentclass{article}
\usepackage[utf8]{inputenc}
\usepackage[MeX,plmath]{polski} 
\usepackage{amsmath}
\usepackage{amsfonts}

\begin{document}
\title{Funkcja błędu Gaussa}
\maketitle

\section*{Wprowadzenie}
Funkcja błędu Gaussa – funkcja nieelementarna, która występuje w rachunku prawdopodobieństwa, statystyce oraz w teorii równań różniczkowych cząstkowych. Jest zdefiniowana jako 
\begin{equation*}
\operatorname {erf} (x)={\frac {2}{\sqrt {\pi }}}\int _{0}^{x}e^{-t^{2}}\,\mathrm {d} t
\end{equation*}

Funkcja erf jest ściśle związana z uzupełniającą funkcją błędu erfc:
\begin{eqnarray*}
\operatorname{erfc}(x) \equiv & 1-\operatorname{erf}(x) & = {\frac{2}{\sqrt{\pi}}}\int _{x}^{\infty }e^{-t^{2}}\,\mathrm {d} t
\end{eqnarray*}

Definiuje się także zespoloną funkcję błędu w(x), nazywaną także funkcją Faddiejewej: 
$$ w(x)=e^{-x^{2}}\operatorname {erfc} \left(-ix\right) $$

\section*{Najważniejsze własności i zastosowania}
Funkcja błędu jest nieparzysta: 
\begin{displaymath}
\operatorname {erf} (z)=-\operatorname {erf} \left(-z\right)
\end{displaymath}

Ponadto należy zauważyć, że prawdziwe jest równanie:
\begin{equation*}
\operatorname {erf} \left(z^{*}\right)=\left(\operatorname {erf} (z)\right)^{*}
\end{equation*}

Dla osi rzeczywistej funkcja błędu przyjmuje następujące granice \( \operatorname {erf} \left(\pm \infty \right)=\pm 1 \), natomiast dla osi urojonej $ \operatorname {erf} \left(\pm i\infty \right)=\pm i\infty $.

Funkcja błędu jest ściśle związana z rozkładem normalnym Gaussa. Można to zauważyć, wyliczając pochodną i funkcję pierwotną funkcji błędu: 
\begin{displaymath}
{\frac {d}{dz}}\operatorname {erf} (z)={\frac {2}{\sqrt {\pi }}}e^{-z^{2}}
\end{displaymath}
$$ F(x)=z\,\operatorname {erf} (z)+{\frac {e^{-z^{2}}}{\sqrt {\pi }}} $$

\section*{Szereg Taylora}
Przez zapisanie prawej strony definicji jako szereg Taylora i całkowanie, można dowieść, że 
\begin{multline*}
\operatorname {erf} (x)={\frac {2}{\sqrt {\pi }}}\sum _{n=0}^{\infty }{\frac {(-1)^{n}x^{2n+1}}{(2n+1)n!}}=\\{\frac {2}{\sqrt {\pi }}}\left(x-{\frac {x^{3}}{3}}+{\frac {x^{5}}{10}}-{\frac {x^{7}}{42}}+{\frac {x^{9}}{216}}-\ \cdots \right)
\end{multline*}

\end{document}