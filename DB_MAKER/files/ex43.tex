\documentclass{article}
\usepackage[utf8]{inputenc}
\usepackage[MeX,plmath]{polski} 
\usepackage{amsmath}
\usepackage{amsfonts}

\begin{document}
\title{Nierówność Gronwalla}
\maketitle

\section*{Wstęp}
Nierówność Gronwalla – jedna z podstawowych nierówności stosowanych w teorii równań różniczkowych zwyczajnych. Stosowana jest m.in. w twierdzeniach o istnieniu i jednoznaczności rozwiązań równań różniczkowych zwyczajnych. Twierdzenie udowodnione po raz pierwszy przez szwedzkiego matematyka, T.H. Grönwalla, w 1918. 

\section*{Nierówność Gronwalla}

Niech (a,b) będzie przedziałem liczb rzeczywistych oraz niech t0 należy do (a,b). Niech ponadto alfa, beta, u będą funkcjami ciągłymi określonymi na (a,b) o wartościach w R+. Jeżeli dla każdego t należącego do (a,b) zachodzi nierówność

\begin{displaymath}
	u(t)\leq \alpha (t)+\left|\int _{t_{0}}^{t}\beta (s)u(s)\,\mathrm {d} s\right|
\end{displaymath}

to dla każdego t należącego do (a,b) zachodzi również

\begin{equation*}
    u(t)\leq \alpha (t)+\left|\int _{t_{0}}^{t}\alpha (s)\beta (s)e^{\left|\int _{s}^{t}\beta (\xi )\,\mathrm {d} \xi \right|}\,{\mbox{d}}s\right|
\end{equation*}

\section*{Dowód}
Poniższy dowód pochodzi od J. A. Oguntuase.

Niech 

\begin{displaymath}
	v(t)=\int _{t_{0}}^{t}\beta (s)u(s){\mbox{d}}s
\end{displaymath}

Wówczas

\begin{equation*}
	v'(t)=\beta (t)u(t)\leq \alpha (t)\beta (t)+\beta (t)\left|\int _{t_{0}}^{t}\beta (s)u(s)\,{\mbox{d}}s\right|=\alpha (t)\beta (t)+\beta (t)\cdot \operatorname {sgn}(t-t_{0})v(t)
\end{equation*}

Ponadto, niech 

\[\gamma (t)=e^{-\int _{t_{0}}^{t}\operatorname {sgn}(s-t_{0})\beta (s)\,\mathrm {d} s}\]

Mnożąc otrzymaną nierówność stronami przez gamma(t) otrzymujemy

\begin{equation}
	\gamma (t)v'(t)\leq \alpha (t)\beta (t)\gamma (t)-v(t)\gamma '(t)
\end{equation}

Ostatecznie,

\begin{displaymath}
	{\frac {\mbox{d}}{{\mbox{d}}t}}(\gamma (t)v(t))-\alpha (t)\beta (t)\gamma (t)\leq 0
\end{displaymath}

Wynika z powyższego, iż

\begin{equation*}
	\operatorname {sgn}(t-t_{0})\int _{t_{0}}^{t}{\frac {\mathrm {d} }{\mathrm {d} t}}(\gamma (s)v(s))-\alpha (s)\beta (s)\gamma (s)\,\mathrm {d} s\leq 0
\end{equation*}

Czyli

\begin{equation}
	\operatorname {sgn}(t-t_{0})\gamma (t)v(t)\leq \operatorname {sgn}(t-t_{0})\int _{t_{0}}^{t}\alpha (s)\beta (s)\gamma (s)\,{\mbox{d}}s
\end{equation}

Ostatecznie,

\[u(t)&\leq \alpha (t)+\left|\int _{t_{0}}^{t}\beta (s)u(s)\,\mathrm {d} s\right|=\ldots=\alpha (t)+\left|\int _{t_{0}}^{t}\alpha (s)\beta (s)e^{\left|\int _{s}^{t}\beta (\xi )\,\mathrm {d} \xi \right|}\,{\mbox{d}}s\right| \]

\end{document}