\documentclass{article}
\usepackage[utf8]{inputenc}
\usepackage[MeX,plmath]{polski} 
\usepackage{amsmath}
\usepackage{amsfonts}

\begin{document}
\title{Twierdzenie Picarda}
\maketitle

\section*{Wprowadzenie}
Twierdzenie Picarda – twierdzenie o istnieniu i jednoznaczności rozwiązań zagadnienia Cauchy’ego. Podawane jest także jako twierdzenie Picarda-Lindelöfa lub twierdzenie Cauchy’ego-Lipschitza. Nazwa twierdzenia ma uhonorować Charlesa Picarda, a w innym wersjach także Ernsta Lindelöfa, Rudolpha Lipschitza i Augustina Cauchy’ego.  

\section*{Twierdzenie}
Załóżmy, że  $\Omega \subseteq {\mathbb {R} }\times {\mathbb {R} }$ jest obszarem otwartym na płaszczyźnie oraz funkcja \(f\colon \Omega \to {\mathbb {R} }\) jest ciągła na zbiorze omega i spełnia warunek Lipschitza ze względu na drugą zmienną. Tak więc, dla pewnej stałej L mamy, że 

\begin{displaymath}
	|f(x,y_{1})-f(x,y_{2})|\leq L|y_{1}-y_{2}|
\end{displaymath}

ilekroć $(x,y_{1}),(x,y_{2})\in \Omega$.

Niech (x0,y0) należy do omegi. Wówczas dla pewnej delty większej od 0, zagadnienie początkowe

\begin{equation*}
	y'=f(x,y)
\end{equation*}

\[y(x_{0})=y_{0}\]

ma dokładnie jedno rozwiązanie $y=\varphi (x)$ określone na przedziale \((x_{0}-\delta ,x_{0}+\delta )\).

\section*{Uogólnienie na przestrzenie Banacha}
Twierdzenie Picarda w naturalny sposób przenosi się na funkcje spełniające lokalny warunek Lipschitza określone na otwartych podzbiorach produktu prostej rzeczywistej i dowolnej przestrzeni Banacha. 

\subsection*{Lokalny warunek Lipschitza}


Niech Y będzie przestrzenią unormowaną oraz  $D\subseteq \mathbb {R} \times Y$ będzie zbiorem otwartym. Mówimy, że funkcja f:D→Y spełnia lokalny warunek Lipschitza na zbiorze D wtedy i tylko wtedy, gdy każdy punkt \((x_{0},u_{0})\in D\) ma otoczenie, na którym f spełnia warunek Lipschitza względem drugiej współrzędnej.

\subsection*{Twierdzenie Picarda}

Niech Y będzie przestrzenią Banacha oraz D należące do R × Y będzie zbiorem otwartym. Jeżeli funkcja f:D→Y jest ciągła oraz spełnia lokalny warunek Lipschitza względem drugiej współrzędnej na zbiorze D, to\\
- każde rozwiązanie równania u'= f(x,u) daje się przedłużyć do rozwiązania globalnego, \\
- każde rozwiązanie globalne powyższego równania jest funkcją określoną na przedziale otwartym, \\
- dla każdego punktu (x0, u0) należącego do D istnieje dokładnie jedno rozwiązanie globalne spełniające warunek początkowy u(x0)=u0.



\section*{Globalne twierdzenie o istnieniu i jednoznaczności rozwiązań}

Korzystając z twierdzenia Picarda można dowieść globalnego twierdzenia o istnieniu i jednoznaczności rozwiązań równań różniczkowych zwyczajnych, znane również jako twierdzenie o istnieniu i jednoznaczności rozwiązania wysyconego. Poza faktem istnienia oraz jedyności rozwiązania opisuje ono również jego zachowanie. 

\end{document}