\documentclass{article}
\usepackage[utf8]{inputenc}
\usepackage[MeX,plmath]{polski} 
\usepackage{amsmath}
\usepackage{amsfonts}

\begin{document}
\title{Funkcja Gudermanna}
\maketitle

\section*{Wprowadzenie}
Funkcja Gudermanna – funkcja specjalna nazwana od imienia niemieckiego matematyka, Christopha Gudermanna, zwana także amplitudą hiperboliczną lub gudermanianem, wyraża się wzorem: 
\begin{displaymath}
{\begin{aligned}{\text{gd }}x&=\int _{0}^{x}{\frac {dt}{\cosh t}}\\&=2{\text{ arctg}}\left({\text{tgh }}{\frac {x}{2}}\right)\\&=2{\text{ arctg }}e^{x}-{\frac {\pi }{2}}\end{aligned}}
\end{displaymath}

\section*{Najważniejsze własności}
Jak widać, stosowane funkcji Gudermanna ukazuje naturalny pomost, jaki istnieje między funkcjami cyklometrycznymi a hiperbolicznymi, bez potrzeby odwoływania się do narzędzi analizy zespolonej.

Zauważmy, że: 
$$ {\text{tgh }}{\frac {x}{2}}={\text{tg }}{\frac {{\text{gd }}x}{2}} $$

Prawdziwe są następujące tożsamości:
\[ \sinh x = \tg({\text{gd }} x) \]
\[ \cosh x = \sec({\text{gd }} x) \]
\[ \tgh x = \sin({\text{gd }} x) \]
\[ {\text{sech }} x = \cos({\text{gd }} x) \]
\[ {\text{csch }} x = \ctg({\text{gd }} x) \]
\[ \ctgh x = \csc({\text{gd }} x) \]

Istnieje sposób wyrażenia funkcji wykładniczej przy użyciu funkcji Gudermanna: 
$$ {\begin{aligned}e^{x}&={\frac {1}{\cos \left({\text{gd }}x\right)}}+{\text{tg}}\left({\text{gd }}x\right)\\&=\sec \left({\text{gd }}x\right)+{\text{tg}}\left({\text{gd }}x\right)\\&={\text{tg}}\left({\frac {\pi }{4}}+{\frac {{\text{gd }}x}{2}}\right)\\&={\frac {1+\sin \left({\text{gd }}x\right)}{\cos \left({\text{gd }}x\right)}}\end{aligned}} $$

Pochodna funkcji Gudermanna wyraża się wzorem: 
\begin{equation*}
{\frac {d}{dx}}\,{\text{gd }}x={\text{sech }}x
\end{equation*}


\end{document}