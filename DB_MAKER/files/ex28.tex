\documentclass{article}
\usepackage[utf8]{inputenc}
\usepackage[MeX,plmath]{polski} 
\usepackage{amsmath}
\usepackage{amsfonts}

\begin{document}
\title{Twierdzenie o rozwijalności funkcji w szereg Fouriera}
\maketitle

\section*{Twierdzenie}
Jeżeli funkcja f(x) jest różniczkowalna w punkcie x0, to jej szereg Fouriera jest zbieżny do wartości funkcji w tym punkcie. Innymi słowy: w punktach różniczkowalności funkcję da się rozwinąć w szereg Fouriera. 

\subsection*{Dowód}
Niech x0 będzie punktem, w którym funkcja f(x) jest różniczkowalna; mamy: 
\begin{displaymath}
{\begin{aligned}&\;a_{n}\cos n\omega x_{0}+b_{n}\sin n\omega x_{0}\\=&\;{\frac {1}{T}}\int \limits _{-T}^{T}f(x)\cos n\omega xdx\cos n\omega x_{0}+{\frac {1}{T}}\int \limits _{-T}^{T}f(x)\sin n\omega xdx\sin n\omega x_{0}\\=&\;{\frac {1}{T}}\int \limits _{-T}^{T}f(x)(\cos n\omega x\cos n\omega x_{0}+\sin n\omega x\sin n\omega x_{0})dx\\=&\;{\frac {1}{T}}\int \limits _{-T}^{T}f(x)\cos n\omega (x-x_{0})dx.\end{aligned}}
\end{displaymath}

Suma cząstkowa szeregu Fouriera przedstawia się w następujący sposób: 
\begin{displaymath}
{\begin{aligned}S_{N}&={\frac {1}{2T}}\int \limits _{-T}^{T}f(x)dx+\sum _{n=1}^{N}{\frac {1}{T}}\int \limits _{-T}^{T}f(x)\cos n\omega (x-x_{0})dx\\&={\frac {1}{T}}\int \limits _{-T}^{T}f(x)\left({\frac {1}{2}}+\sum _{n=1}^{N}\cos n\omega (x-x_{0})\right)dx.\end{aligned}}
\end{displaymath}

Stosując do tego wyrażenia lemat II, otrzymujemy następujący wzór: 
\begin{displaymath}
S_{N}={\frac {1}{T}}\int \limits _{-T}^{T}f(x){\frac {\sin \omega \left(N+{\frac {1}{2}}\right)(x-x_{0})}{2\sin \omega {\frac {1}{2}}(x-x_{0})}}dx
\end{displaymath}

Funkcja podcałkowa w powyższym wzorze jest funkcją o okresie 2T, możemy więc dokonać przesunięcia w dziedzinie i otrzymujemy: 
\begin{displaymath}
S_{N}={\frac {1}{T}}\int \limits _{-T}^{T}f(x+x_{0}){\frac {\sin \omega \left(N+{\frac {1}{2}}\right)x}{2\sin \omega {\frac {1}{2}}x}}dx
\end{displaymath}

Funkcja tożsamościowo równa 1 na całym zbiorze liczb rzeczywistych jest rozwijalna w szereg Fouriera w każdym punkcie, kładąc f(x) = 1, mamy: 
\begin{displaymath}
1={\frac {1}{T}}\int \limits _{-T}^{T}{\frac {\sin \omega \left(N+{\frac {1}{2}}\right)x}{2\sin \omega {\frac {1}{2}}x}}dx
\end{displaymath}

Mnożąc powyższą równość przez f(x0) i odejmując obustronnie od równania przedstawiającego sumę cząstkową szeregu, otrzymujemy: 
\begin{displaymath}
S_{N}-f(x_{0})={\frac {1}{T}}\int \limits _{-T}^{T}(f(x+x_{0})-f(x_{0})){\frac {\sin \omega \left(N+{\frac {1}{2}}\right)x}{2\sin \omega {\frac {1}{2}}x}}dx
\end{displaymath}

Rozważmy następującą granicę: 
\begin{displaymath}
\lim _{x\to 0}{\frac {f(x+x_{0})-f(x_{0})}{2\sin \omega {\frac {1}{2}}x}}=\lim _{x\to 0}{\frac {f(x+x_{0})-f(x_{0})}{x}}{\frac {x}{2\sin \omega {\frac {1}{2}}x}}={\frac {f'(x_{0})}{\omega }}
\end{displaymath}
przy obliczaniu której korzystamy z różniczkowalności funkcji f(x) w punkcie x0.

Możemy określić następującą funkcję: 
\begin{displaymath}
\phi (x)=\left\{{\begin{aligned}&{\frac {f(x+x_{0})-f(x_{0})}{2\sin \omega {\frac {1}{2}}x}}&&{\text{gdy}}~~x\neq 0\\&{\frac {f'(x_{0})}{\omega }}&&{\text{gdy}}~~x=0\end{aligned}}\right.
\end{displaymath}

Mając na uwadze fakt, iż zmiana skończonej ilości wartości funkcji podcałkowej nie wpływa na wartość całki, wzór (2) możemy zapisać w postaci: 
\begin{displaymath}
S_{N}-f(x_{0})={\frac {1}{T}}\int \limits _{-T}^{T}\phi (x)\sin \omega \left(N+{\frac {1}{2}}\right)xdx
\end{displaymath}

Funkcja podcałkowa spełnia założenia lematu Riemanna, tak więc: 
\begin{displaymath}
\lim _{N\to \infty }(S_{N}-f(x_{0}))=\lim _{N\to \infty }{\frac {1}{T}}\int \limits _{-T}^{T}\phi (x)\sin \omega \left(N+{\frac {1}{2}}\right)xdx=0
\end{displaymath}

q.e.d.

\end{document}