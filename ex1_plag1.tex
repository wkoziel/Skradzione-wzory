\documentclass{article}
\usepackage[utf8]{inputenc}
%Wzory zamienione kolejnością
\begin{document}
 Integer feugiat, lacus ut placerat tincidunt, metus arcu lobortis nibh, non consectetur tortor lectus nec risus. Curabitur varius leo velit, ut maximus odio commodo ut. Maecenas odio lacus, pharetra a magna sed, ornare euismod turpis. Vestibulum in feugiat erat, eget venenatis augue. Mauris at convallis orci, ut mattis eros. Aliquam volutpat ante vel lobortis efficitur. Phasellus vitae ullamcorper lectus.
$$
\lim_{n \to \infty}
\sum_{k=1}^n \frac{1}{k^2}
= \frac{\pi^2}{6}
$$
Sed tortor ipsum, volutpat a eros non, tristique suscipit leo. Cras sagittis porttitor metus, nec feugiat ipsum ultrices vel. Donec ornare dolor semper mollis lacinia. Praesent vel magna nulla. Aliquam iaculis semper rutrum. Fusce dictum, nibh quis laoreet convallis, lorem sem posuere velit, dictum volutpat purus leo eget nibh. Nulla in dolor a lacus scelerisque volutpat at ac libero. Donec cursus dignissim ultrices. Cras eu tellus nec sem mollis elementum nec id dui.
\(
(a+b)^{2}=a^{2}+2ab+b^{2}
\)
Nunc hendrerit magna nisi, quis posuere felis congue a. Nam vestibulum, nulla sit amet sodales consectetur, ligula enim congue justo, at euismod purus justo et justo. Pellentesque iaculis tortor sed enim tristique, vel luctus arcu facilisis. Duis luctus mauris lacus, id pharetra risus pellentesque a. Integer feugiat, lacus ut placerat tincidunt, metus arcu lobortis nibh, non consectetur tortor lectus nec risus. Curabitur varius leo velit, ut maximus odio commodo ut. Maecenas odio lacus, pharetra a magna sed, ornare euismod turpis. Vestibulum in feugiat erat, eget venenatis augue. Mauris at convallis orci, ut mattis eros. Aliquam volutpat ante vel lobortis efficitur. Phasellus vitae ullamcorper lectus.$(a-b)^{2}=a^{2}-2ab+b^{2} $

Sed euismod accumsan justo, ac ullamcorper urna pellentesque ac. Aliquam vel quam rhoncus leo malesuada aliquam viverra eu felis. Curabitur sed convallis nunc. Nullam eu faucibus magna, ut feugiat urna. Aliquam at leo at erat iaculis interdum. Maecenas id ullamcorper mauris. Nullam id eros lectus. Vestibulum ante ipsum primis in faucibus orci luctus et ultrices posuere cubilia curae; Nullam dapibus eu leo et rhoncus. Integer at dolor leo. Donec ultricies mi in iaculis eleifend. Sed blandit sed nisl quis facilisis. Cras at nisi semper, ornare ipsum sed, mattis orci. Donec et urna sit amet purus finibus porta. In sit amet varius augue. Phasellus volutpat bibendum metus, iaculis fermentum dolor consequat ac.
$$ a^{3} - b^{3} = (a-b)(a^{2}+ab+b^{2}) $$

Mauris venenatis velit erat, vitae lobortis magna tincidunt in. Suspendisse nec consectetur elit, vel feugiat enim. Pellentesque ut ante metus. Fusce mauris erat, ultricies quis posuere non, pellentesque nec odio. Aliquam dignissim erat vel dui vestibulum, ut tempus dui mattis. Nunc eu orci vitae orci ultricies congue eget ut nisl. Cras vel molestie nisl. In aliquet, magna at faucibus sollicitudin, dui sem facilisis nibh, et congue tortor urna at sapien.
\[ a^{3} + b^{3} = (a+b)(a^{2}-ab+b^{2}) \]
Mauris venenatis velit erat, vitae lobortis magna tincidunt in. Suspendisse nec consectetur elit, vel feugiat enim. Pellentesque ut ante metus. Fusce mauris erat, ultricies quis posuere non, pellentesque nec odio. Aliquam dignissim erat vel dui vestibulum, ut tempus dui mattis. Nunc eu orci vitae orci ultricies congue eget ut nisl. Cras vel molestie nisl. In aliquet, magna at faucibus sollicitudin, dui sem facilisis nibh, et congue tortor urna at sapien.
Nunc hendrerit magna nisi, quis posuere felis congue a.
\begin{math}
    a^{2}-b^{2} = (a - b)(a + b)
\end{math}
Nam vestibulum, nulla sit amet sodales consectetur, ligula enim congue justo, at euismod purus justo et justo. Pellentesque iaculis tortor sed enim tristique, vel luctus arcu facilisis. Duis luctus mauris lacus, id pharetra risus pellentesque a. Integer feugiat, lacus ut placerat tincidunt, metus arcu lobortis nibh, non consectetur tortor lectus nec risus. Curabitur varius leo velit, ut maximus odio commodo ut. Maecenas odio lacus, pharetra a magna sed, ornare euismod turpis. Vestibulum in feugiat erat, eget venenatis augue. Mauris at convallis orci, ut mattis eros. Aliquam volutpat ante vel lobortis efficitur. Phasellus vitae ullamcorper lectus.
\begin{equation}
(a-b)^{2}=a^{2}-2ab+b^{2} 
\end{equation}
Duis laoreet, justo in fringilla egestas, orci enim placerat arcu, venenatis auctor justo quam in lacus. Nam risus metus, efficitur in vulputate ac, blandit sed nulla. Vivamus id vestibulum eros, vel scelerisque urna. Vivamus gravida semper ex nec pharetra. Vivamus eu finibus est, non accumsan leo. Mauris finibus interdum erat venenatis porta. 
\begin{displaymath}
a^{3} + b^{3} = (a+b)(a+b)(a+b)(a+b)
\end{displaymath}

Lorem ipsum dolor sit amet, consectetur adipiscing elit. Maecenas ut pretium risus. Nunc lectus justo, volutpat a sapien quis, ullamcorper porta dolor. Curabitur feugiat finibus urna nec dapibus. Praesent eget mattis purus. Suspendisse potenti. Curabitur et molestie metus. Donec massa nisi, rhoncus sit amet massa ac, aliquet scelerisque nisi. Pellentesque ut ipsum vitae nisi vulputate mollis. Duis in magna sed lectus sodales feugiat.

Mauris venenatis velit erat, vitae lobortis magna tincidunt in. Suspendisse nec consectetur elit, vel feugiat enim. Pellentesque ut ante metus. Fusce mauris erat, ultricies quis posuere non, pellentesque nec odio. Aliquam dignissim erat vel dui vestibulum, ut tempus dui mattis. Nunc eu orci vitae orci ultricies congue eget ut nisl. Cras vel molestie nisl. In aliquet, magna at faucibus sollicitudin, dui sem facilisis nibh, et congue tortor urna at sapien.
Nunc hendrerit magna nisi, quis posuere felis congue a. Nam vestibulum, nulla sit amet sodales consectetur, ligula enim congue justo, at euismod purus justo et justo. Pellentesque iaculis tortor sed enim tristique, vel luctus arcu facilisis. Duis luctus mauris lacus, id pharetra risus pellentesque a.
$\lim_{n \to \infty}
\sum_{k=1}^n \frac{1}{k^2}
= \frac{\pi^2}{6}$

 Integer feugiat, lacus ut placerat tincidunt, metus arcu lobortis nibh, non consectetur tortor lectus nec risus. Curabitur varius leo velit, ut maximus odio commodo ut. Maecenas odio lacus, pharetra a magna sed, ornare euismod turpis. Vestibulum in feugiat erat, eget venenatis augue. Mauris at convallis orci, ut mattis eros. Aliquam volutpat ante vel lobortis efficitur. Phasellus vitae ullamcorper lectus.
\begin{math}
    a^{a}-b^{a} = (b - a)(a + b)
\end{math}
\end{document}
